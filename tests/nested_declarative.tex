\begin{theorem}[Name 1]
    \begin{example}[Name 2]
        \[
            a = 1
        \]
        \[
            b = 1
        \]
    \end{example}

    \begin{proof}
        \[
            c = 1
        \]
    \end{proof}
\end{theorem}

% Should Give
% `````{prf:theorem}
% ````{prf:example}
%
% ````
%
% `````

\begin{example}

    \noindent
    What is the the fraction of uncompensated ions on each side of the
    membrane required to produce 100 mV in a spherical cell with
    a radius of 25 $\mu m$?

    Assumptions: the membrane capacitance is $1 \mu F/cm^2$ and the concentration of ions inside and outside is $0.5$ M (mol/L)
    The number of ions needed to charge up 1 $cm^2$ membrane
    to 100 mV is
    \[
    \frac{q \cdot 1 (cm^2)}{e} = \frac{10^{-6}(C V^{-1} cm^{-2}) \cdot
    10^{-1} (V) \cdot (cm^2)}{1.6 \cdot 10^{-19}(C)} = 6 \cdot 10^{11} .
    \]
    Thus, the total number of uncompensated ions needed for the
    cell membrane is
    \[
    6 \cdot 10^{11} (cm^{-2}) \cdot 4 \pi (0.0025)^2 (cm^2) =
    4.7 \cdot 10^7 ,
    \]
    where the second term on the LHS is the surface area of the sphere
    with 25 $\mu m$ radius.
    Since the total number of ions in the volume of the spherical cell is
    \[
    0.5 ~(mol/L) \cdot 6.02 \cdot 10^{23} / (mol)
    \cdot \frac{4}{3} \pi (0.0025)^3 \cdot 10^{-3} (L) = 2\cdot 10^{13},
    \]
    the fraction of uncompensated ions amounts to
    $4.7 \cdot 10^7 / 2 \cdot 10^{13}$ or 0.000235\%.
\end{example}


\begin{theorem}{\textbf{(SISO Multidimensional CIM)}}
    \label{thm:general_cim}\\
    Let {$\{u_n^{\,i}\,|\,u_n^{\,i}\in\mathcal{H}_{n}\}_{i=1}^N$} be a collection of $N$ linearly independent
    stimuli at the input to a [Filter]-[Leaky IAF] circuit with a spatiotemporal receptive field {$h_n \in H_{n}$.}
        If $N\geq(2L_1+1)\cdot ... \cdot (2L_{n-1}+1)$ and the neuron produces at least $2L_n+2$ spikes per signal, then
        the filter projection $\mathcal{P}h_n$ can be perfectly identified from a collection of I/O pairs
        {$\{(u_n^{\,i},\mathbb{T}^{\,i})\}_{i=1}^N$} as
        \[
            (\mathcal{P}h_n)(x_1,...,x_{n-1},t) = \sum_{|l_1|\leq L_1} ... \sum_{|l_n|\leq L_n}h_{l_1 l_2 ... l_n}\, e_{l_1 l_2 ... l_n}(x_1, ... ,x_{n-1},t),
        \]
        where $\mathbf{h} = \mathbf{\Phi}^+\mathbf{q}$. Furthermore,
        $\mathbf{\Phi}=[\mathbf{\Phi}^1;\,\mathbf{\Phi}^2;\,...\,\, ;\mathbf{\Phi}^N]$ and $\mathbf{q}=[\mathbf{q}^1;\,\mathbf{q}^2;\,...\,\, ;\mathbf{q}^N]$, $[\mathbf{q}^i]_k = q_k^i$.
        The elements of each matrix $\mathbf{\Phi}^i$ are given by
        \begin{IEEEeqnarray}{rCl}
            [\mathbf{\Phi}^i]_{kl}  = \frac{RCL_n\sqrt{T_n}u^i_{-l_1,...,-l_{n-1},l_n}}{jl_n\Omega_n RC+L_n}\left[e_{l_n}(t_{k+1}^i)-e_{l_n}(t_{k}^i)\exp\left(\frac{t_k^i-t_{k+1}^i}{RC}\right)\right]
        \end{IEEEeqnarray}
        with the column index
        {$l$ traversing all possible subscript combinations of $l_1, l_2, ..., l_n$ for all $k\in\mathbb{Z}$, $i=1,2,...\,,N$.}
        %$l=(l_1+L_1)(2L_2+1)...(2L_n+1)+(l_2+L_2)...(2L_n+1)+...+l_n+L_n+1$,  for all $k\in\mathbb{Z}$, $l_1=-L_1,-L_1+1,...\,,L_1$, $l_2=-L_2,-L_2+1,...\,,L_2$,..., $l_n=-L_n,-L_n+1,...\,,L_n$ and $i=1,2,...\,,N$.
\end{theorem}